% vim:ts=4:sw=4:expandtab
% © 2011 Michael Stapelberg
%
% use xelatex %<
%
\documentclass[xetex,serif,compress]{beamer}
\usepackage{fontspec}
\usepackage{xunicode} % Unicode extras!
\usepackage{xltxtra}  % Fixes
\usepackage{listings}
\setmainfont{Trebuchet MS}
\setmonofont{Inconsolata}
\usetheme{default}

\setbeamertemplate{frametitle}{
    \color{black}
    \vspace*{0.5cm}
    \hspace*{0.25cm}
    \textbf{\insertframetitle}
    \par
}

% Hide the navigation icons at the bottom of the page
\setbeamertemplate{navigation symbols}{}

\begin{document}

% slide with bullet points
\newcommand{\mslide}[2]{
    \begin{frame}{#1}
        \begin{list}{$\bullet$}{\itemsep=1em}
            #2
        \end{list}
    \end{frame}
}

\frame{
\begin{center}
\vspace{1.5cm}
{\huge systemd}\\
{\large ein init-replacement}\\
\vspace{2cm}
c¼h von sECuRE\\
\vspace{0.5cm}
NoName e.V., 2011-08-18\\
\vspace{0.5cm}
powered by \LaTeX
\end{center}
}

\begin{frame}{}
\begin{center}
{\Huge Teil 1: Was ist init?}
\end{center}
\end{frame}

\mslide{init}{
	\item PID 1, vom Kernel gestartet
	\item startet daemons, inklusive X11 bzw. xdm
	\item adoptiert elternlose Prozesse (fork),\\
    startet sie ggf. neu (z.B. getty)
}

\begin{frame}[fragile]{SysV init Beispiel: /etc/init.d/thinkfan (1/9)}
\begin{verbatim}
#! /bin/sh
### BEGIN INIT INFO
# Provides:          thinkfan
# Required-Start:    $remote_fs
# Required-Stop:     $remote_fs
# Default-Start:     2 3 4 5
# Default-Stop:      0 1 6
# Short-Description: thinkfan initscript
# Description:       starts thinkfan if enabled in /etc/default/thinkfan
### END INIT INFO

# Author: Evgeni Golov <evgeni@debian.org>

# Do NOT "set -e"
\end{verbatim}
\end{frame}

\begin{frame}[fragile]{systemd-analyze}
    \begin{figure}
    \includegraphics[width=0.5\textwidth]{plot.pdf}
    \caption{\texttt{systemd-analyze plot > plot.svg}}
    \end{figure}
\end{frame}

\end{document}
